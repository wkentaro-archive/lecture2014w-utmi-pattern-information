\section{チャレンジ課題3}\label{section:challenge3}
\begin{itembox}{チャレンジ課題3}
  表情認識を行うプログラムを作成せよ.データセットには日本人女性の顔のデータセット(JAFFE データセッ ト)を用いよ.JAFFE データセットには 10 名の被験者から得られた 213 枚の画像が含まれ,7 つの表情(幸福, 悲しみ,驚き,怒り,不快,恐れ,無表情)を行っている.
  このデータセットで,10 名のうち 9 名の画像を用いて学習を行った後,残りの 1 名に対して表情識別テストを 行え.この試行を 10 名に対して繰り返すことで 10 名分の平均識別率を求めよ.画像からの特徴量,表情の識別 手法は各自好きなものを用いて良い.もちろん独自に勉強した手法も大歓迎.7 つの表情を全て識別させるのは難 しいので,幸福と悲しみの 2 クラスだけでもよい.
\end{itembox}

特徴抽出をHOG, Autoencoder~\cite{autoencoder}を用いて行い,
識別器はSVM~\cite{sklearn-svm}とロジスティック回帰を利用した.
特徴抽出と識別率は表~\ref{tbl:feature-extraction-score}のようになった.

\begin{table}[htbp]
  \begin{center}
    \begin{tabular}{ccc}
      特徴抽出 & 識別器 & 識別率 \\
      None & SVM & 0.5812836 \\
      HOG & SVM & 0.5887888  \\
      Autoencoder & SVM & 0.2878157 \\
      HOG+Autoencoder & SVM & 0.4934857  \\
      None & ロジスティック回帰 & 0.6695595 \\
      HOG & ロジスティック回帰 & 0.6126256 \\
    \end{tabular}
    \caption{特徴抽出, 識別器と識別率}
    \label{tbl:feature-extraction-score}
  \end{center}
\end{table}
