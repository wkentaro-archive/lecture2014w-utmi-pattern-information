%        File: pattern_report2014.tex
%     Created: Thu Jan 22 03:00 AM 2015 J
% Last Change: Thu Jan 22 03:00 AM 2015 J
%
\documentclass[10pt,a4paper,twocolumn]{jarticle}

%%%%%%%%%%%%%%%%%%%%%
% to input Japanese %
%%%%%%%%%%%%%%%%%%%%%
\usepackage[japanese]{babel}

%%%%%%%%%%%
% unknown %
%%%%%%%%%%%
\usepackage{ascmac}

%%%%%%%%%%%%%%%%%%%%%%%%%%
% to be standard a4paper %
%%%%%%%%%%%%%%%%%%%%%%%%%%
\usepackage{geometry}
\geometry{
  a4paper,
  total={210mm,297mm},
  left=20mm,
  right=20mm,
  top=20mm,
  bottom=40mm,
}

%%%%%%%%%%%%%%%%%%%%%
% to insert figures %
%%%%%%%%%%%%%%%%%%%%%
\usepackage[dvipdfmx]{graphicx}

%%%%%%%%%%%%%%%%%%%%%%%%%%
% to insert source codes %
%%%%%%%%%%%%%%%%%%%%%%%%%%
% \usepackage{listings, jlisting}
% \renewcommand{\lstlistingname}{list}
% \lstset{language=C,
%   basicstyle=\ttfamily\scriptsize,
%   commentstyle=\textit,
%   classoffset=1,
%   keywordstyle=\bfseries,
%   frame=tRBl,
%   framesep=5pt,
%   showstringspaces=false,
%   numbers=left,
%   stepnumber=1,
%   numberstyle=\tiny,
%   tabsize=2
% }

%%%%%%%%%%%%%%%%%%
% title & author %
%%%%%%%%%%%%%%%%%%
\title{パターン情報学 プログラミングレポート課題}
\author{03-140299 東京大学機械情報工学科 3年 和田健太郎}

%%%%%%%%%%%%%%%%%%
% begin document %
%%%%%%%%%%%%%%%%%%
\begin{document}
\maketitle

\section{課題1}
\begin{itembox}[1]{課題1}
2 クラス($\omega$1,$\omega$2)の識別問題を考える.データは 2 次元とする.配布するデータセットの説明を以下に示す.

\begin{itemize}
  \item Train1.txt,Train2.txt:$\omega$1,$\omega$2 に属する訓練データ集合.各データ数 50.
  \item Test1.txt,Test2.txt: $\omega$1,$\omega$2 に属するテストデータ集合.各データ数 20.
\end{itemize}

2 クラスで,2 次元のデータに対するウィドロー・ホフのアルゴリズムを実装し,訓練データから分離超平面を
学習せよ.また,テストデータの識別率(全テストデータ数に対する正しく識別されたテストデータ数の比率)を
求めよ.さらに,訓練データ,テストデータ,学習された識別面を図示せよ.
\end{itembox}


%%%%%%%%%%%%%%%%%%%%%%%%%%
% to insert bibliography %
%%%%%%%%%%%%%%%%%%%%%%%%%%
% \begin{thebibliography}{9}
%   \bibitem{inv1} Samuel R.Buss,"Introduction to Inverse Kinematics with Jacobian Transpose,Pseudoinverse and Damped Least Squares methods"
% \end{thebibliography}

%%%%%%%%%%%%%%%%
% end document %
%%%%%%%%%%%%%%%%
\end{document}

